\documentclass[review]{elsarticle}\usepackage[]{graphicx}\usepackage[]{color}
%% maxwidth is the original width if it is less than linewidth
%% otherwise use linewidth (to make sure the graphics do not exceed the margin)
\makeatletter
\def\maxwidth{ %
  \ifdim\Gin@nat@width>\linewidth
    \linewidth
  \else
    \Gin@nat@width
  \fi
}
\makeatother

\definecolor{fgcolor}{rgb}{0.345, 0.345, 0.345}
\newcommand{\hlnum}[1]{\textcolor[rgb]{0.686,0.059,0.569}{#1}}%
\newcommand{\hlstr}[1]{\textcolor[rgb]{0.192,0.494,0.8}{#1}}%
\newcommand{\hlcom}[1]{\textcolor[rgb]{0.678,0.584,0.686}{\textit{#1}}}%
\newcommand{\hlopt}[1]{\textcolor[rgb]{0,0,0}{#1}}%
\newcommand{\hlstd}[1]{\textcolor[rgb]{0.345,0.345,0.345}{#1}}%
\newcommand{\hlkwa}[1]{\textcolor[rgb]{0.161,0.373,0.58}{\textbf{#1}}}%
\newcommand{\hlkwb}[1]{\textcolor[rgb]{0.69,0.353,0.396}{#1}}%
\newcommand{\hlkwc}[1]{\textcolor[rgb]{0.333,0.667,0.333}{#1}}%
\newcommand{\hlkwd}[1]{\textcolor[rgb]{0.737,0.353,0.396}{\textbf{#1}}}%
\let\hlipl\hlkwb

\usepackage{framed}
\makeatletter
\newenvironment{kframe}{%
 \def\at@end@of@kframe{}%
 \ifinner\ifhmode%
  \def\at@end@of@kframe{\end{minipage}}%
  \begin{minipage}{\columnwidth}%
 \fi\fi%
 \def\FrameCommand##1{\hskip\@totalleftmargin \hskip-\fboxsep
 \colorbox{shadecolor}{##1}\hskip-\fboxsep
     % There is no \\@totalrightmargin, so:
     \hskip-\linewidth \hskip-\@totalleftmargin \hskip\columnwidth}%
 \MakeFramed {\advance\hsize-\width
   \@totalleftmargin\z@ \linewidth\hsize
   \@setminipage}}%
 {\par\unskip\endMakeFramed%
 \at@end@of@kframe}
\makeatother

\definecolor{shadecolor}{rgb}{.97, .97, .97}
\definecolor{messagecolor}{rgb}{0, 0, 0}
\definecolor{warningcolor}{rgb}{1, 0, 1}
\definecolor{errorcolor}{rgb}{1, 0, 0}
\newenvironment{knitrout}{}{} % an empty environment to be redefined in TeX

\usepackage{alltt}
\makeatletter
\def\ps@pprintTitle{%
 \let\@oddhead\@empty
 \let\@evenhead\@empty
 \def\@oddfoot{}%
 \let\@evenfoot\@oddfoot}
\makeatother
% \usepackage{graphicx}
% \usepackage{mathtools}
% \usepackage{float}
% \usepackage{appendix}
% \usepackage{authblk}
% \usepackage{setspace}
% \usepackage{fullpage}
% \usepackage{natbib}
% \usepackage{lineno}

\title{Psychological stress and cortisol responses: Somatic anxiety is negatively related to cortisol change}

\author[jsc]{J.S.~Chrabaszcz\corref{cor1}}
    \ead{jchrabaszcz@gmail.com}
\author[pjm]{P.J.~Moore}
    \ead{pjmoore@gwu.edu}
\author[ck]{C.E.~Kennedy}
\ead{}
\author[me]{M.H.~Eisenberg Colman}
\ead{}
\author[wc]{W.D.~Charmak}
\ead{}

\address[jsc]{Software Engineering Institute, Carnegie Mellon University}
\address[pjm]{Department of Psychology, The George Washington University}
\address[ck]{Product Team, 98point6 Inc.}
\address[me]{Communications Research Division, Fors Marsh Group}
\address[wc]{Kimbrough Ambulatory Care Clinic, United States Department of Defense}

\cortext[cor1]{Corresponding Author}
\IfFileExists{upquote.sty}{\usepackage{upquote}}{}
\begin{document}

\maketitle



\begin{abstract}

Our goal is to further elucidate the relationship between psychological and physiological responses to stressors.
Better understanding of this relationship could provide important clues to limiting the impact of stress and anxiety on physical, emotional and behavioral outcomes.
In two studies --- an initial study and a subsequent replication --- we examined young adults' self-reported anxiety and salivary cortisol levels  before and after they viewed a painful blood-draw scenario.
In both studies, we find significant increases in participants' post-scenario anxiety, no reliable change in cortisol levels, and a negative correlation between pre-post anxiety and cortisol change-scores.
While counter to many previous studies relating social anxiety to cortisol levels, these results are generally consistent with prior research on cortisol and somatic stress.
These findings suggest that neuroendocrine responses to stress may depend, at least in part, on the psychological nature of that stress.

\end{abstract}

\begin{keyword}
stress \sep anxiety \sep measurement \sep cortisol
\end{keyword}


\section*{Introduction}
\label{sec:introduction}

Stress can have devastating effects on human health and well-being, resulting in emotional disturbances \cite{almeida1998everyday}, memory deficits \cite{vedhara2000acute}, impaired personal relationships \cite{gruber2003long}, and weakened immune function \cite{o1990stress}.
These effects can also harm friends, family, and others who interact with individuals under stress \cite{mulia2008stress}.
In addition, stress has tremendous economic impact, costing an estimated \$42 billion a year in the United States alone \cite{greenberg1999economic}.

Among the most prominent forms of stress is anxiety, which directly affects over 32 million Americans who suffer from one or more anxiety disorders \cite{kessler2005prevalence}.
As a result, anxiety has been identified as one of the world's most important, ongoing epidemics \cite{jessop2004brief}.
Given the enormous costs of stress---and anxiety in particular---it is important to develop a fuller understanding of their mechanism(s).
A central component of this process is the activation of the hypothalamic-pituitary-adrenal (HPA) axis and the release of stress-related hormones, including cortisol.
Psychological stress can cause the hypothalamus to release corticotropin-releasing hormone (CRH), which stimulates the anterior pituitary gland to secrete adenocorticotrophic hormone (ACTH) into the blood, triggering the release of cortisol from the adrenal glands.

Another method of assessing the impact of stress is to obtain people's self-reports of anxiety, either in general \cite{adkins2008psychometric} or in response to potentially stressful events \cite{leininger2012cortisol}.
While cortisol and other physiological measures are less subject to response bias, they are not emotion-specific.
Conversely, though self-reports are more prone to demand characteristics, then can provide qualitative, emotion-specific information about the stress experience, including anxiety.
Because the cognitive process leading to anxiety can activate a neuroendocrine cascade resulting in cortisol release, cortisol is often utilized as a positive marker for self-reported anxiety \cite{kirschbaum1993trier}.
This positive anxiety-cortisol link is supported by a substantial body of research \cite{kelly2008sex,boudarene2001study}. 
However, a number of factors suggest that the strength---and even direction---of anxiety's relationship with cortisol may depend on the type and intensity of the anxiety.

Although studies on the anxiety-cortisol link have included a variety of stressors, the majority of this research has focused on social anxiety \cite{campbell2012acute}.
Previous research has identified empirical, psychological distinctions between anxiety based on social concerns (e.g., evaluation, embarrassment), and anxiety about somatic threats, including pain, injury and/or death \cite{taylor1999anxiety}.
There are also indications that this distinction may lead to different cortisol responses \cite{turner2010sex}.
For example, in a comprehensive review of cortisol responses to stress among children and adolescents, \cite{gunnar2009stressor} found that 12 out of 17 public-speaking (i.e., social-stress) studies showed significant elevations in cortisol levels (with 2 showing decreases in cortisol and 3 finding no significant differences).
However, in the 18 studies of somatic stress (e.g., heights, scary movies, medical diagnoses, venipuncture) among children and adolescents, 14 studies reported no significant cortisol changes, while 4 found decreases in cortisol, and none reported cortisol increases.
Whether these results generalize to emerging adulthood---a period of significant stress---is unknown, for no studies to date have examined the link between somatic anxiety and cortisol levels in this population.

Additional factors may also limit the ability of many anxiety-cortisol studies to assess the relationship between anxiety and cortisol responses.
For example, many studies that measure cortisol levels do not explicitly compare cortisol to self-reported anxiety \cite{steudte2011decreased}.
Still other studies have assessed this association purely in terms of pre-post mean differences within each measure, rather than examining correlated changes between them \cite{kudielka2009human}.
However, significant intra-measure change is not required for correlational significance, which can exist in the absence of one or more mean changes.
Moreover, correlational studies of anxiety and cortisol typically examine this relationship using nominal levels of each measure before or after a stressor.
To the extent that self-reports and physiological measures are intended as markers for an individual's stress response, it is important to assess the relationship between stress-related \emph{changes} in self-reported anxiety and objective cortisol levels \cite{blood1994spouses,harrell1996situational,walco2005procedural}.

To address these issues, the current research assessed the association between the changes in young adults' self-reported anxiety and salivary cortisol levels before and after being presented with a vivid written description of a physically painful event.
We also conducted a partial replication with an added video presentation to assess both the reliability of the initial results and the impact of additional stress on the strength of the anxiety-cortisol relationship.
Using this repeated-measure design, we tested the following hypotheses:

\begin{enumerate}
  \item That participants' self-reported anxiety will increase significantly after viewing the physically painful event;
  \item That participants' salivary cortisol will not increase significantly after being presented with the description of the physically painful event; and,
  \item That pre-post changes in anxiety will be negatively correlated with pre-post changes in salivary cortisol levels.
\end{enumerate}

\begin{center}
\section*{Study 1}
\label{sec:study1}
\end{center}

\section*{Method}
\label{sec:method1}

\subsection*{Participants}
\label{sub:participants1}

Thirty-eight undergraduates from a mid-sized urban university participated in partial fulfillment of course requirements.
Twenty-five (66\%) of the participants were female and an equal number were Caucasian.
Participants ranged in age from 17 to 30 years old, with an average age of 19.

\subsection*{Procedure}
\label{sub:procedure1}

Participants were recruited through the Psychology Department subject pool.
After arriving for their scheduled session and giving informed consent, each participant provided an initial salivary cortisol sample and completed an anxiety questionnaire.
Research sessions for all participants were scheduled between 9am and 12pm to reduce variability associated with normal diurnal cortisol rhythms.
Participants were then presented with a written scenario depicting a very painful event (described below), and they were asked to respond as if they themselves were in this situation.
After reading the scenario, participants completed a second anxiety questionnaire and then---approximately 15 minutes after viewing the painful scenario---a second salivary cortisol sample was taken.
Finally, after providing demographic information (i.e., age, sex, ethnicity), participants were debriefed about the background and purpose of the study and were thanked for their participation.

\subsection*{Painful event scenario}
\label{sub:paineventscenario}

The painful event scenario presented to participants read as follows:

\begin{quote}
``You are at a medical clinic awaiting a physical examination required by your health insurance carrier.
Because of repeated delays, this is the last opportunity for you take the physical before your policy expires.
This examination includes having your blood drawn to determine your red and white blood cell counts, as well as your iron and serum cholesterol levels.
The blood is drawn by medical technicians, who usually take it from a vein at the base of the forearm, opposite the elbow.
Almost all of the technicians associated with this clinic are highly skilled, experienced professionals.
However, the newest technician has yet to master the technique of drawing blood.
As a result, he typically misses the vein on the first attempt, requiring him to move the needle around inside the arm, and often necessitating repeated insertions.
Because there are only ten technicians currently on duty, the odds of having your blood drawn by the new technician is one out of ten, or 10\%.
Having blood drawn in this manner is extremely painful, including a sharp pain when the needle punctures the skin, and even more intense pain as the needle is angled inside the arm in search of a vein. In addition to being extremely painful, this procedure often results in temporary nerve damage causing, for up to two weeks, a continuous burning sensation, much like having a burning match under your arm.''
\end{quote}

This scenario has proven to be an effective means of increasing anxiety ratings in previous research \cite{moore2010cognitive,kennedy2011makes}, and pilot participants for the current study exhibited a ceiling effect for anxiety at objective probabilities of 10\% or higher.

\subsection*{Measures}
\label{sub:measures}

\emph{Anxiety}. Participant anxiety was assessed using state version of Spielberger's 10-item state-trait anxiety inventory (STAI), which is designed to assess individuals' anxiety-related feelings and sensations \cite{spielberger1983assessment}.
The STAI has been found to be a valid and reliable index of anxiety \cite{bados2010state,marteau1992development}.
In completing the STAI before and after viewing the painful event scenario, participants indicated the extent to which they were experiencing each feeling/sensation at that moment (e.g., ``I feel jittery'') using a scale from 1 (not at all) to 4 (very much so), with positively-phrased items reverse coded so that higher values reflected greater anxiety.
Responses were then combined and averaged to create an overall anxiety score for each participant ($\alpha = $ 0.87).

\emph{Cortisol}. Participants' cortisol was obtained through saliva samples, a recommended method for cortisol collection \cite{salimetrics2011saliva}.
While 85-99\% of cortisol in blood is bound to serum proteins, the vast majority of salivary cortisol remains unbound \cite{vining1983hormones}.
In addition, salivary cortisol levels are unaffected by salivary enzymes or flow rates \cite{vining1987measurement}.
To minimize contamination and maintain proper oral pH balance, only participants who reported not having eaten, drunk, or brushed their teeth in the previous 60 minutes were included in the study.
Saliva samples were collected using oral cotton swabs placed under participants' tongues for approximately 90 seconds, after which each swab was inserted into a plastic storage tube, sealed, and stored in a freezer at $-23^{\circ}$ Celsius within 45 minutes of collection.
Within four weeks after being collected, all saliva samples were shipped to Salimetrics, LLC (State College, PA) for analysis.

Salivary cortisol levels ($\mu$g/dL) were assessed using a competitive enzyme immunoassay.
Cortisol antibodies from the saliva were incubated in the presence of their antigen.
Following incubation, available antibody/antigen complexes were added to an antigen-coated well plate where the complexes were conjugated to an enzyme (horseradish peroxidase).
Well plates were washed (removing any unbound antigen and conjugate) and a substrate (tetramethylbenzidine) was added to react with the horseradish peroxidase for visualization.
The optical density of the produced color is inversely proportional to the concentration of salivary cortisol.

\emph{Statistical Analyses}. After preliminary analyses were conducted, a paired t-test was used to compare anxiety scores before and after participants viewed the painful-event scenario (Hypothesis 1). 
A paired t-test was also used to compare participants' pre- and post-scenario salivary cortisol levels (Hypothesis 2).
To test Hypothesis 3, change-scores for both anxiety and salivary cortisol were calculated by subtracting pre-scenario levels of each variable from their corresponding post-scenario levels.
The relationship between these change-scores for anxiety and cortisol was then assessed using a Pearson product-moment correlation.

To control for both undue outlier influence and overestimation of effect sizes, we also analyzed the change score correlations using Bayesian parameter estimation.
We modeled normalized change in cortisol in the first experiment as a linear function of the normalized change in STAI score with $t$-distributed errors.
We allowed the degrees-of-freedom parameter of the error distribution to vary, with a prior belief that lower, heavier-tailed values of the parameter more likely \cite{juarez2010model}.
The Bayesian equivalent of robust regression, this reduces the potential for outlying observations to exert a disproportionate effect on parameter estimates \cite{kruschke2010doing}.
We also included a double exponential prior on the effect of STAI change centered at zero with a scale parameter of one.
This is the Bayesian equivalent of a LASSO procedure \cite{tibshirani1996regression}, and is designed to protect against overestimation of effect sizes \cite{gelman2013bayesian}.
This can be interpreted as an \emph{a priori} assumption that any effect would be close to zero and that a single standard deviation change in anxiety would typically predict less than a single standard deviation change in cortisol.
These simulations were conducted in R version 3.4.3 \cite{rmanual2014} and Stan version 2.17.2 \cite{stan-software:2014}.

\section*{Results}
\label{sec:results1}

A total of five saliva samples were unsuitable for analysis, two pre-manipulation and three post-manipulation.
Each unsuitable sample came from a unique participant.
These participants were excluded from analysis involving cortisol, resulting in a reduced sample size for some comparisons.
None of the tests that do not include cortisol are substantially changed by omitting these five participants.

\emph{Pre-manipulation}. Participants' anxiety prior to viewing the painful-event scenario averaged 0.04 (\textit{SD} = 0.05), and did not differ as a function of sex, age, or ethnicity ($p \mbox{s} > .26$).
Cortisol levels, also unrelated to sex and ethnicity ($p \mbox{s} > .11$), averaged 2.03 $\mu$g/dL (\textit{SD} = 0.64).
Cortisol levels were not significantly related to age, (\textit{t}(32) = -1.83, \textit{p} =  0.08).

\emph{Post-manipulation}. Participants' anxiety after viewing the scenario averaged 0.43 (\textit{SD} = 0.17), and did not differ across sex, age or ethnicity ($p \mbox{s} > .19$).
Average cortisol levels, again unrelated to sex and ethnicity ($p \mbox{s} > .18$), were 0.39 $\mu$g/dL (\textit{SD} = 0.17).
As before, cortisol levels after the manipulation were negatively, though not significantly, related to age (\textit{t}(32) = -1.89, \textit{p} = 0.07).

\emph{Pre-post comparisons}. Participants' anxiety increased significantly after viewing the painful-event scenario (\textit{t}(33) = 3.44, \textit{p} = 0.002), while their cortisol levels remained unchanged (\textit{t}(33$) = $ 4.17, \textit{p} = 0.0002053).
As shown in Figure \ref{fig:scatter}, changes in participants' pain-related anxiety were significantly (and inversely) related to changes in their salivary cortisol levels (\textit{r} = -0.39, \textit{t}(32) = -2.41, \textit{p} = 0.02).
This was confirmed with Bayesian parameter estimation, despite using a robust $t$ distribution of the errors and shrinking the effect size estimate toward zero, (Table \ref{tab:bayes1}).
No changes in cortisol or STAI were significantly related to demographic variables ($p \mbox{s} > .16$).

\begin{table}[ht]
\centering
  \caption{Posterior estimates for modeled parameters in Study 1 including empirical estimates of the 95\% Bayesian interval around each parameter. Effective N and $\hat{R}$ are convergence diagnostics for the markov chain process.}
  \begin{tabular}{rrrrrrr}
  \hline
  Parameter & Mean & St. Dev. & 2.5\% & 97.5\% & Effective N & $\hat{R}$ \\
% latex table generated in R 3.4.3 by xtable 1.8-2 package
% Thu Jan 11 15:24:43 2018
  \hline
Intercept & 0.05 & 0.01 & 0.031 & 0.071 & 3,474 &   1 \\ 
  Slope & -0.04 & 0.019 & -0.076 & -0.0034 & 3,634 &   1 \\ 
  Residual SD & 0.05 & 0.0072 & 0.038 & 0.066 & 3,329 &   1 \\ 
   \hline

  \end{tabular}
  \label{tab:bayes1}
\end{table}

\section*{Discussion}
\label{sec:discussion1}

Participants' self-reported anxiety increased significantly after viewing the painful-event scenario in the current study.
This result is consistent with a large body of research showing significant effects of stressful scenarios on stress responses, including pain-related anxiety \cite{chung2005pain,moore2009information}.
However, participants' salivary cortisol levels remained unchanged after viewing the scenario.
While counter to research on social anxiety and cortisol, this finding is generally consistent with cortisol results from studies involving somatic-anxiety studies \cite{turner2010sex,noto2005relationship}.

The significant, inverse relationship between pre-post changes in participants' self-reported anxiety and salivary cortisol levels is inconsistent with previous, positive links between general anxiety and cortisol change-scores \cite{alpers2003salivary,sachar1970cortisol}.
However, the current study appears to be the first correlational test of change-scores for cortisol levels and pain-related anxiety.
To assess the reliability of these findings, a replication of this study was conducted. 
Given prior research indicating that the strength of stress-related anxiety-cortisol associations may depend on the intensity of the stressor, a graphic blood-draw video was added the blood-draw scenario to enhance the intensity of the stress manipulation.

\begin{center}
\section*{Study 2}
\label{sec:study2}
\end{center}

\section*{Method}
\label{sec:method2}

\subsection*{Participants}
\label{sub:participants2}

Forty undergraduates from the the same university participated in partial fulfillment of course requirements.
Thirty (75\%) of participants were female, and 26 (65\%) were Caucasians.
Participants ranged in age from 18 to 21 years old, with an average age of 19.
All participants were new to the study and naive to the manipulation.

\subsection*{Procedure}
\label{sub:procedure2}

The procedure for Study 2 was identical to Study 1, except for the addition of a blood-draw video.
A video approximately one minute in length depicting a close-up shot of four vials of blood being drawn from a human arm was displayed on a laptop screen for participants to watch prior to reading the vignette used in Study.

\subsection*{Measures}
\label{sub:measures2}

\emph{Anxiety}. Participants' anxiety was again assessed using Spielberger's 10-item STAI state-trait anxiety inventory \cite{spielberger1983assessment}, with positively-phrased items reverse-coded so that higher values reflected greater anxiety.
As in Study 1, responses were then combined and averaged to create an overall self-reported anxiety score for each participant ($\alpha =$ 0.75).

\emph{Cortisol}. Participants' cortisol was obtained through saliva samples, using identical procedures of salivary cortisol collection, storage, and quantification as in Study 1.

\emph{Statistical Analyses}. The same statistical analyses (i.e., paired and two-sample t-tests and pre-post change-score correlation) were used to test the same three hypotheses as in Study 1.

\section*{Results}
\label{sec:results2}

One participant contributed no usable saliva samples.
This individual was omitted from all analyses.

\emph{Pre-manipulation}. Participants' anxiety prior to viewing the blood-drawing scenario and video averaged -0.01 (\textit{SD} = 0.13), and did not differ as a function of sex, age or ethnicity ($p \mbox{s} > .29$).
The average pre-manipulation cortisol level was 2.13 $\mu$g/dL (\textit{SD} =c0.62).
While unrelated to age or ethnicity, male participants' cortisol levels ($\mbox{M} = $ 0.56, \textit{SD} = 0.26) were significantly higher than those of female participants ($\mbox{M} = $ 0.33, \textit{SD} = 0.21, \textit{t}(37$) = $ 2.11, \textit{p} = 0.04).

\emph{Post-manipulation}. As seen in Table \ref{tab:descriptives}, participants' anxiety after the manipulation averaged 0.39 (\textit{SD} = 0.24), and did not differ across sex, age or ethnicity ($p \mbox{s} > .29$).
Participants' average post-manipulation cortisol level was 0.4 $\mu$g/dL (\textit{SD} = 0.25), with males($\mbox{M} = $ 0.54, \textit{SD} = 0.23) exhibiting higher levels than females ($\mbox{M} = $ 0.36, \textit{SD} = 0.24, \textit{t}(37) = 2.83, \textit{p} = 0.01).
Post-manipulation cortisol levels were unrelated to any other demographic variables ($p\mbox{s} > .44$).

\emph{Pre-post comparisons}. Participants' anxiety increased significantly after viewing the blood-draw scenario and video (\textit{t}(38) = 4.85, \textit{p} = 0), while salivary cortisol levels remained unchanged (\textit{t}(38) = 0.59, \textit{p} = 0.56).
Changes in participants' pain-related anxiety were negatively, though not significantly, related to changes in their salivary cortisol levels (\textit{r} = -0.31, \textit{t}( 37) = -1.98, \textit{p} = 0.06; Figure \ref{fig:scatter}).
No changes in cortisol or STAI were significantly related to demographic variables ($p \mbox{s} > .16$).

\begin{table}[ht]
\centering
\caption{Descriptive statistics for STAI and salivary cortisol measurements split by collection period.}
\begin{tabular}{rrrrrrrr}
\hline
& & & Pre & & Post & & Post - Pre\\ [0.5ex]
Study 1 & & & Mean (SD) & & Mean (SD) & & Mean (SD)\\
\hline
& STAI & & 0.04 (0.05) & & 0.43 (0.17) & & 1.73 (0.52)\\
& Cortisol & & 2.03 (0.64) & & 0.39 (0.17) & & 0.29 (0.5)\\
Study 2 & & & & & & &\\
\hline
& STAI & & -0.01 (0.13) & & 0.39 (0.24) & & 1.68 (0.42)\\
& Cortisol & & 2.13 (0.62) & & 0.4 (0.25) & & 0.44 (0.57)\\
\hline
\end{tabular}
\label{tab:descriptives}
\end{table}

\begin{figure}[ht]
\begin{center}
\caption{Scatterplots with overlaid regression lines relating change in salivary cortisol and change in STAI score, faceted by study.}
\label{fig:scatter}

\begin{knitrout}
\definecolor{shadecolor}{rgb}{0.969, 0.969, 0.969}\color{fgcolor}
\includegraphics[width=\maxwidth]{figure/scatter-1} 

\end{knitrout}

\end{center}
\end{figure}

To assess the combined results of the current studies, manipulation effect sizes were aggregated across studies for each of the dependent variables.
The combined results for Studies 1 \& 2 are summarized in Table \ref{tab:effectsize}.
The overall effect sizes (expressed in terms of Pearson's $r$) on participants' anxiety, cortisol levels, and change-score correlations were, respectively, \textit{r} = 0.55, (\textit{p} =  $5.511 \times 10^{-7}$), \textit{r} = 0.88, (\textit{p} = $2.948 \times 10^{-25}$), \textit{r} = -0.34, (\textit{p} = 0.003728$)$.

The results of our Bayesian analysis are concordant with those from null hypothesis significance testing.
The effect of change in change in anxiety on change in cortisol is reliably negative, as zero is not included in the 95\% high-density interval when aggregating over a prior intended to shrink the effect size estimate and the data from both studies (Table \ref{tab:bayes2}).

\begin{table}[ht]
\centering
\caption{Correlations between pre and post STAI and cortisol measurements and change scores for Study 1, Study 2, and combined.}
\begin{tabular}{rrrrr}
\hline
 Study & & Anxiety & Cortisol & Change \\ [0.5ex]
 \hline
  1 & & \textit{r} = 0.65, \textit{p} = $2.898 \times 10^{-5}$ & \textit{r} = 0.95, \textit{p} = $4.667 \times 10^{-18}$ & \textit{r} = -0.39, \textit{p} = 0.022 \\
  2 & & \textit{r} = 0.48, \textit{p} = 0.002 & \textit{r} = 0.86, \textit{p} = $1.315 \times 10^{-12}$ & \textit{r} = -0.31, \textit{p} = 0.055 \\ [0.5ex]
 \hline
  Combined & & \textit{r} = 0.55, \textit{p} = $5.511 \times 10^{-7}$ & \textit{r} = 0.88, \textit{p} = $2.948 \times 10^{-25}$ & \textit{r} = -0.34, \textit{p} = 0.004 \\
 [1ex]
 \hline
\end{tabular}
\label{tab:effectsize}
\end{table}

\begin{table}[ht]
\centering
  \caption{Posterior estimates for modeled parameters in Study 2, labels carried over from Study 1. These posteriors use the distribution on beta from Study 1 as prior information.}
  \begin{tabular}{rrrrrrr}
  \hline
  Parameter & Mean & St. Dev. & 2.5\% & 97.5\% & Effective N & $\hat{R}$ \\
% latex table generated in R 3.4.3 by xtable 1.8-2 package
% Thu Jan 11 15:24:44 2018
  \hline
Intercept & 0.0068 & 0.02 & -0.033 & 0.046 & 2,254 &   1 \\ 
  Slope & -0.04 & 0.00034 & -0.041 & -0.04 & 4,000 &   1 \\ 
  Residual SD & 0.12 & 0.017 & 0.087 & 0.15 & 2,138 &   1 \\ 
   \hline

  \end{tabular}
  \label{tab:bayes2}
\end{table}

% section results2 (end)

We could also fit a multilevel model to these data in which the average level of cortisol change, and the effect of STAI change on cortisol change, vary by study.

% \begin{table}[ht]
% \centering
% \caption{}
% \begin{tabular}{}

% \end{tabular}
% \end{table}

\begin{knitrout}
\definecolor{shadecolor}{rgb}{0.969, 0.969, 0.969}\color{fgcolor}\begin{kframe}
\begin{alltt}
\hlkwd{print}\hlstd{(fit.3,} \hlkwc{digits} \hlstd{=} \hlnum{2}\hlstd{)}
\end{alltt}
\begin{verbatim}
## stan_lmer
##  family:       gaussian [identity]
##  formula:      cortchange ~ staichange + (1 + staichange | study)
##  observations: 73
## ------
##             Median MAD_SD
## (Intercept)  0.04   0.02 
## staichange  -0.06   0.03 
## sigma        0.10   0.01 
## 
## Error terms:
##  Groups   Name        Std.Dev. Corr
##  study    (Intercept) 0.0611       
##           staichange  0.0577   0.08
##  Residual             0.0967       
## Num. levels: study 2 
## 
## Sample avg. posterior predictive distribution of y:
##          Median MAD_SD
## mean_PPD 0.01   0.02  
## 
## ------
## For info on the priors used see help('prior_summary.stanreg').
\end{verbatim}
\end{kframe}
\end{knitrout}

\begin{knitrout}
\definecolor{shadecolor}{rgb}{0.969, 0.969, 0.969}\color{fgcolor}
\includegraphics[width=\maxwidth]{figure/predictplot-1} 

\end{knitrout}


\section*{Discussion} % (fold)
\label{sec:discussion2}

As in Study 1, Study 2 participants' anxiety ratings increased significantly after the pain-related manipulation.
Although the pre-post change in anxiety was nominally higher in Study 2—--perhaps due to the additional blood-draw video—--this difference was not statistically significant ($p > .42$).
Also consistent with Study 1, participants' pre- and post-manipulation cortisol levels were virtually identical, although unlike in Study 1, male participants in Study 2 exhibited higher cortisol levels than females, both before and after the manipulation.
These differences were not accompanied by gender differences in anxiety and did not interact with the manipulation ($p > .46$), suggesting that the males in our sample had higher salivary cortisol unrelated to their affective states.
Pre-post changes in participants' anxiety and cortisol levels were negatively correlated, although this relationship was not as strong as in Study 1, a difference that may be attributable to greater cortisol variability in Study 2.
With the combined results of Studies 1 and 2, post-manipulation anxiety ratings were significantly higher than pre-manipulation levels and these changes in anxiety were significantly and negatively related to changes in cortisol.

% section discussion2 (end)

% section study2 (end)

\section*{General Discussion} % (fold)
\label{sec:discussiong}

In both current studies, three consistent patterns emerged.
First, participants' anxiety ratings increased significantly after viewing a physically painful event.
Supporting the first hypothesis, these results are also consistent with an extensive literature demonstrating significant effects of experimental stress manipulations on individuals' self-reported anxiety \cite{turner2010sex,gunnar2009stressor,harrell1996situational,walco2005procedural}.
The pre-post increase in anxiety ratings in Study 2, while also significant, was not significantly greater than the anxiety increase in Study 1. This suggests that the blood-draw video did not add significantly to stress associated with the original written scenario; alternatively, it may reflect a ceiling effect for indirect stress manipulations.

Second, pre- and post-manipulation cortisol levels in both studies were virtually identical.
It may be that cortisol levels are more resistant to increase than self-rated anxiety, suggesting the need for a stronger stress manipulation.
These results, however, are largely consistent with previous somatic-anxiety studies, many of which involved the direct threat or actual experience of pain \cite{king2010anxiety,ashman2002stress,battaglia1997physiological,quas2004physiological}.
It is also possible that a stress-related increase in cortisol was countered by the general, downward trend in cortisol levels during the day.
This too seems unlikely, for while cortisol levels decrease significantly between 9am and 9pm \cite{legler1982diurnal}, cortisol samples were taken from each participant within 20 minutes of each other. While cortisol levels typically rise (and then fall) significantly within an hour of a large meal, only prospective participants who reported not having consumed anything at least 60 minutes before the study were included in this research.
Of course, null results must be interpreted cautiously, for there are many other potential explanations for such findings, including random cortisol fluxuations, measurement error, and shipping and handling problems (e.g., contamination, decay).
That said, in the context of pain-related anxiety, although cortisol increases have been reported \cite{alpers2003salivary}, they appear to be the rare exception rather than the rule.

Finally, across the two current studies, pre-post increases in self-reported anxiety were associated with corresponding decreases in salivary cortisol.
Consistent with the third hypothesis, these results also demonstrate that significant associations between change-scores do not require significant changes in the relevant factors.
They also suggest that another stress-related mechanism, other than through the HPA axis, may predominate in the case of somatic anxiety; specifically, a pathway that leads to a decrease in cortisol.
One possible pathway may be through oxytocin, which prevents the release of cortisol by reducing ACTH and is itself released during painful events \cite{tops2007anxiety,neumann1999brain,gibbs1986oxytocin}.

Stimulation of the paraventricular nucleus---the trigger for CRH and eventually cortisol release---also causes the release of oxytocin directly into the posterior pituitary \cite{ross2009characterization}.
Oxytocin release can in turn cause reciprocal down-regulation of neural activity in the hypothalamus for both the paraventricular and supraoptic nuclei, which could suppress CRH release and subsequent cortisol expression.
If oxytocin, which is implicated in fear responses \cite{guzman2013fear}, is released proportional to expected pain during these potentially stressful events, it could block cortisol release and cause the observed, negatively correlated changes between self-reported anxiety and salivary cortisol levels without affecting the mean levels of cortisol.
This would be consistent with non-human studies showing differing patterns of hormone response to qualitatively different stressors \cite{andersen2004different}.
Future research assessing both cortisol and oxytocin may provide a better understanding of the mechanism translating anxiety into physiological stress.
Although not novel, correlational analysis of pre-post anxiety and cortisol change scores is relatively rare.
One notable exception is \cite{sachar1970cortisol}, who found a positive relationship between general anxiety and cortisol changes during and after psychiatric patients' depressive episodes (both anxiety and cortisol levels were lower after patients' recovery).
The current studies appear to be the only research to directly assess the relationship between simultaneous changes in pain-related anxiety and cortisol levels.

Limitations of this research include the scenario-based nature of the research design, which may have limited the impact of the stress manipulation.
Participants may also have experienced stress en route to the experiment, erroneously increasing the pre-test measurement of salivary cortisol levels.
However, while these factors may help explain the unchanged cortisol levels, they would neither explain the observed increases in pain-related anxiety ratings, nor the significant relationship between anxiety and cortisol change-scores.
In addition, many previous studies have successfully increased participants' anxiety ratings by having them view stressful situation, including those involving pain \cite{kennedy2011makes,moore2010cognitive}.
Nonetheless, it would be useful for future research to examine the anxiety-cortisol link in response to the threat of actual pain in a controlled fashion (e.g, cold-pressor, mild electric shock).
Given the current findings, one would expect greater increases in anxiety, and a stronger, negative relationship between changes in anxiety and cortisol,
The future use of multiple cortisol measurements before and after the stress manipulation may also reduce measurement error, making the link between anxiety and cortisol differences easier to detect.
Another limitation is the potential bias associated with self-reported anxiety.
In the current research, participants' anxiety ratings may reflect what participants thought was expected of them, rather than their affective experience.
Howver, while this may be an alternative explanation for the observed post-manipulation increase in anxiety ratings, it would not explain the negative relationship between anxiety and cortisol change-scores.
Nonetheless, this issue could be addressed in future research by assessing social desirability, a measure designed to control for the extent to which individuals wish to be seen by others in a positive light.

In sum, the current research includes the first correlational test of somatic anxiety and cortisol changes.
The results are broadly consistent with previous research, and together they suggest that social and somatic anxiety may elicit different---perhaps even opposing---neuroendocrine responses.
Reflecting the complexity of the psychological and physiological effects of stress, current and future research on the link between these responses provide a key to understanding the impact of stressful experences on human health and well-being.

% section discussiong (end)

\section*{Declarations}

The authors declare that there is no conflict of interest.
Protocol for this study were approved by the Institutional Review Board at the George Washington University, protocol \# 050722, dated June 13, 2007.

\pagebreak

\bibliographystyle{apalike}
\bibliography{fesc}

\end{document}
